%Based on RTU EEF guidelines for bachelor thesis
%https://www.rtu.lv/writable/public_files/RTU_025.pdf
%Modified for smaller tasks

\documentclass[12pt,fleqn,titlepage,oneside]{article}

%Page margins according to guidelines
\usepackage[a4paper,top=30mm,right=15mm,bottom=30mm,left=35mm]{geometry}
%Indent first paragraph after section header
\usepackage{indentfirst}
%Set space between lines (1, 1.5, 2)
\usepackage{setspace}
%Set paragraph first line indent according to guidelines
\setlength{\parindent}{15mm}

%Select alternative section titles
\usepackage{titlesec}
%Double line spacing after section title
\titlespacing*{\section}
	{0pt}{\baselineskip}{2\baselineskip}
%Normal spacing around subsection title
\titlespacing*{\subsection}
	{0pt}{\baselineskip}{\baselineskip}

%Change enumerate number formatting
\usepackage{enumitem}
%More enumeration options
\usepackage{moreenum}

%Set table of contents depth to subsections
\setcounter{tocdepth}{2}
%Control table of contents, figures, etc
\usepackage{tocloft}
%Add " lpp." after section page number in table of contents
\renewcommand{\cftsecafterpnum}{\textbf{ lpp.}}
%Add " lpp." after subsection page number in table of contents
\renewcommand{\cftsubsecafterpnum}{ lpp.}

%Improved interface for floating objects
\usepackage{float}
%Support for sub-captions
\usepackage{subcaption}
%Produces figures which text can flow around
\usepackage{wrapfig}
%Control float placement with \FloatBarrier
\usepackage{placeins}

%AMS mathematical facilities for LaTeX
\usepackage{amsmath}
%Mathematical tools to use with amsmath
\usepackage{mathtools}
%Typeset physical units following the rules of the International System of Units (SI)
\usepackage{SIunits}
%Code formatting
\usepackage{listings}

%Tabulars with adjustable-width columns
\usepackage{tabularx}
%Driver-independent color extensions for LaTeX and pdfLaTeX
\usepackage{xcolor}
%Enhanced support for graphics (\includegraphics)
\usepackage{graphicx}

%Create PostScript and PDF graphics in TeX
\usepackage{tikz}
%Draw electrical networks with TikZ
\usepackage[EFvoltages]{circuitikz}
%Easy generation of timing diagrams as TikZ pictures
\usepackage{tikz-timing}
%Create normal/logarithmic plots in two and three dimensions
\usepackage{pgfplots}
\pgfplotsset{compat=newest}

%Accept different input encodings
\usepackage[utf8]{inputenc}

%Add section number to equations, figures and tables
\numberwithin{equation}{section}
\numberwithin{figure}{section}
\numberwithin{table}{section}

%Set caption label names according to guidelines
\renewcommand{\contentsname}{Saturs}
\renewcommand{\figurename}{att.}
\renewcommand{\tablename}{tabula}
%Set caption label format according to guidelines
\DeclareCaptionLabelFormat{tableLabelFormat}{#2 #1}
\DeclareCaptionLabelFormat{figureLabelFormat}{#2 #1}
\DeclareCaptionLabelSeparator{figureLabelSeperator}{ }
\DeclareCaptionFormat{mine}{\raggedleft #1\\\centering #3}%
\captionsetup[table]{labelformat=tableLabelFormat, singlelinecheck=off, format=mine}
\captionsetup[figure]{labelsep=figureLabelSeperator, labelformat=figureLabelFormat}

%Used for drawing a dashed vertical line from a point to the x axis in pgfplots
\newcommand{\vertLineFromPoint}[1]{
  \draw[dashed] 
  (#1) -- (#1|-{rel axis cs:0,0})
}
%Used for drawing a dashed horizontal line  from a point to the y axis in pgfplots
\newcommand{\horLineFromPoint}[1]{
  \draw[dashed] 
  (#1) -- (#1-|{rel axis cs:0,0})
}

%Variables used in the document
\newcommand{\authorName}{Valters Melnalksnis}
\newcommand{\authorId}{161REC035}

%Bibliography
\usepackage[language=latvian]{biblatex}
\BiblatexLatvianWarningOff
\addbibresource{bibliography.bib}

%PDF hyperlinks, navigation, etc.
\usepackage{url}
\usepackage{hyperref}
\hypersetup{
	breaklinks=true,
	bookmarks=true,
	unicode=true,
	hidelinks
}

%Split document in multiple files
\usepackage{subfiles}

%Render to external files to minimize compile time
\usetikzlibrary{external}
\tikzset{external/optimize=false}
\usepgfplotslibrary{external}
\tikzexternalize

\begin{document}
\begin{titlepage}
	\centering
	\doublespacing
	{\Large \textbf{Rīgas Tehniskā Universitāte}}
	
	{Elektrotehnikas un vides inženierzinātņu fakultāte\\}
	{Industriālās elektronikas un elektrotehnikas institūts\\}
	{Industriālās elektronikas un elektrotehnoloģiju katedra\par}
	\vspace{2cm}
	
	\onehalfspacing
	{\Large \textbf{\authorName}\\}
	Profesionālā bakalaura adaptronikas studiju programmas students,\\
	stud. apl. nr. \authorId
	\vspace{2cm}
	
	{\huge 18650 tipa bateriju elementu ietilpības pētīšanas iekārtas izpēte un izstrāde\par}
	
	\doublespacing
	{\Large Bakalaura darbs ar projekta daļu\par}
	\onehalfspacing
	\vspace{6cm}
	
	\raggedleft
	Zinātniskais vadītājs\\
	RTU IEEI pētnieks\\
	Kristaps Vītols
	
	\centering
	\vfill
	Rīga, \the\year
\end{titlepage}

\onehalfspacing
\FloatBarrier
\newpage
\setcounter{page}{2}
\section*{\MakeUppercase{Anotācija}}

\newpage

\section*{\MakeUppercase{Abstract}}

\newpage

\doublespacing
\tableofcontents
\onehalfspacing
\newpage

\addcontentsline{toc}{section}{\texorpdfstring{\MakeUppercase{Ievads}}{Ievads}}
\section*{\texorpdfstring{\MakeUppercase{Ievads}}{Ievads}}

\FloatBarrier
\newpage
\section{\texorpdfstring{\MakeUppercase{Teorijas izpēte}}{Teorijas izpēte}}

\subsection{Baterijas}

Trīs funkcionālās litija jonu bateriju sastāvdaļas ir elektrolīts un pozitīvais un negatīvais elektrods.
Parasti negatīvais elektrods tiek izgatavots no oglekļa, bet pozitīvais - metāla oksīda.
Elektrolīts ir litija sāls organiskā šķidinātājā.

18650 (vai arī 168A) šūnu tipiskā ietilpība ir no 1500 līdz 3500 $mAH$, nominālais spriegums 3.7 $V$,
diametrs 18 $mm$ un garums 65 $mm$.\cite{18650cell}
Lai gan fiziskie izmēri ir standartizēti, šūnām ar izvirzītu pozitīvo termināli var atšķirties izvirzījuma izmēri,
līdz ar to tās var būt nesaderīgas ar ierīcēm.

\subsubsection{Uzlāde}

Vienas šūnas uzlādei ir divi posmi - konstantas strāvas un konstanta sprieguma.
Šie posmi veidojas no diviem šūnas parametriem - maksimālās uzlādes strāvas un maksimālā sprieguma.

Uzlādējot vairākas virknē slēgtas šūnas vēl nepieciešams balansēt uzlādi starp šūnām, 
jo tās var uzlādēties dažādos ātrumos atšķirīgu iekšējo pretestību dēļ.

\subsubsection{Izlāde}

\subsection{Līdzīgas iekārtas}

\FloatBarrier
\newpage
\section{\texorpdfstring{\MakeUppercase{Sistēmas analīze/apraksts}}{Sistēmas analīze/apraksts}}

\subsection{Izlādes modulis}

\subsection{Uzlādes modulis}

\subsection{Vadības modulis}

\FloatBarrier
\newpage
\section{\texorpdfstring{\MakeUppercase{Secinājumi}}{Secinājumi}}

\FloatBarrier
\newpage

\printbibliography[
	heading=bibintoc,
	title=\texorpdfstring{\MakeUppercase{Izmantotā literatūra}}{Izmantotā literatūra}
]

\end{document}
